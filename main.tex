\documentclass{article}
\usepackage[utf8]{inputenc}
\usepackage{graphicx}
\usepackage{amsmath}
\usepackage{hyperref}
\usepackage{natbib}
\usepackage{geometry}
\geometry{a4paper, margin=1in}

\title{A Mixed-Methods Study on the Transition to Microservices Architecture}
\author{Pouya Ataei}
\date{21st November 2023}

\begin{document}

\maketitle

\section{Introduction}
\label{sec:introduction}

This research aims to study the transition from a monolithic system to a microservices architecture within a company (IDEXX in this case), employing a mixed-methods approach integrating the Technology Acceptance Model (TAM) and statistical analysis.

\section{Motivation}
\label{sec:motivation}

The motivation behind this study is to understand the dynamics of technology adoption within organizations, particularly focusing on the transition to microservices. It seeks to contribute to the broader field of information systems by providing empirical insights into acceptance and usage patterns of new technological frameworks.

\section{Study Duration}
\label{sec:duration}

The proposed study will span over a period of 6 to 12 months, allowing for comprehensive data collection and analysis across various stages of the transition.

\section{Target Journals}
\label{sec:targetJournals}

The research aims to target high-impact journals in the fields of information systems and software engineering, including:
\begin{itemize}
    \item MIS Quarterly
    \item Proceedings of the IEEE
    \item ACM Computing Surveys
    \item Journal of Management Information Systems
    \item Information Systems Research
    \item Journal of Software: Evolution and Process
\end{itemize}

\section{Research Design}
\label{sec:researchDesign}

The research will employ a mixed-methods approach, starting with qualitative methods like interviews and observations to understand the initial transition process. This will be followed by a quantitative phase, utilizing surveys based on the TAM framework. Data will be analyzed using statistical methods, including regression analysis and Structural Equation Modeling (SEM). Additionally, AI techniques like Natural Language Processing (NLP) will be applied to qualitative data for sentiment analysis.

The items below provide a sequential step for the research: 

\begin{enumerate}
    \item \textbf{Initial Qualitative Phase:}
    \begin{itemize}
        \item Conducting interviews and observations to gather qualitative data about the transition to microservices architecture.
    \end{itemize}
    
    \item \textbf{Quantitative Phase:}
    \begin{itemize}
        \item Implementing surveys based on the Technology Acceptance Model (TAM) to quantitatively assess aspects related to the acceptance and use of microservices architecture.
    \end{itemize}
    
    \item \textbf{Data Analysis:}
    \begin{itemize}
        \item Applying AI techniques like Natural Language Processing (NLP) for the analysis of qualitative data.
        \item Using statistical methods such as regression analysis and Structural Equation Modeling (SEM) for quantitative data analysis.
    \end{itemize}
    
    \item \textbf{Final Synthesis:}
    \begin{itemize}
        \item Combining insights from both qualitative and quantitative phases to provide a comprehensive understanding of the research topic.
    \end{itemize}
\end{enumerate}

\section{Conclusion}
\label{sec:conclusion}

This research design offers a comprehensive approach to studying the adoption of microservices architecture, combining qualitative insights with quantitative rigor. The findings are expected to contribute valuable knowledge to the academic community and practical implications for the industry, aligning with the standards of top-tier academic journals.

\section{Funding}
\label{sec:funding}

This research does not require substantial financial support, estimated between 5 to 10 thousand USD. The funding is necessary to cover a range of costs including advanced analytical tools, software licenses, publication fees, and other operational expenses associated with a study of this scale and complexity. Securing this funding is crucial for the successful execution and dissemination of the research findings.

\end{document}

